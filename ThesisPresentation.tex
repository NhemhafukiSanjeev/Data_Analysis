\documentclass{beamer}
\setbeamertemplate{navigation symbols}{\insertbiblabel}
\setbeamertemplate{bibliography item}[text]
\setbeamertemplate{footline}{
	\begin{beamercolorbox}[wd=\paperwidth, sep=0.1cm]{footline}
		\hspace*{0.1cm}
		\hfill
		\insertframenumber
		/
		\inserttotalframenumber
		\hspace{0.1cm}
	\end{beamercolorbox}
}
\graphicspath{{Images/}{./}} 

\usepackage{booktabs} 
\usetheme{Frankfurt}

\usefonttheme{serif} 
\usepackage{palatino} 
\usepackage{opensans}
\usepackage{ragged2e}

\useinnertheme{rounded}

\usepackage{graphicx}
\usepackage{pdfpages}
\usepackage{multirow}
\usepackage{xcolor} % package for colored citations
% Define a custom command for colored citations
\newcommand{\coloredcite}[2]{\textcolor{blue}{\cite{#2}}}

\BeforeBeginEnvironment{Specific Objectives:}{
	\setbeamercolor{block title example}{fg=white,bg=grey!50!white}
	\setbeamercolor{block body example}{fg=cyan, bg=cyan!30!white}
}
\AfterEndEnvironment{example}{
	\setbeamercolor{block title example}{fg=exblocktitlefgsave,bg=exblocktitlebgsave}
	\setbeamercolor{block body example}{fg=exblockbodyfgsave,bg=exblockbodybgsave}
}

\title[Short Title]{Household Vulnerability and Environmental Dependency in Rural Nepal} 
\author[Sanjeev Nhemhafuki]{Sanjeev Nhemhafuki*} 
\institute[CEDECONTU]{Central Department of Economics \\ \smallskip {Tribhuvan University}} 
\date[\today]{\\
	\vspace{0.7cm}
	\scriptsize \today}

\begin{document}
	{
		\setbeamercolor{background canvas}{bg=}
		\includegraphics[width=0.86\paperwidth,height=0.86\paperheight]{front.pdf}
	}
	\begin{frame}[plain]
		\titlepage  
	\end{frame}
	
	\begin{frame}
		\frametitle{Table of Contents}
		\tableofcontents % Table of contents slide
	\end{frame}
	
	
	\section{Background of the study}
	\begin{frame}[t]{Background of the study}
		\begin{itemize}
			\item \begin{justify} 
				\small Global rural population \small\coloredcite{}{r1} comprises 43.5\% of the total population, with a declining trend, but is projected to be 68\% urban by \small\coloredcite{}{r2} 2050.
			\end{justify} 
			
			\item \begin{justify}
				\small The rural population of 3.4 billion \small\coloredcite{}{r3} is vulnerable, with 80\% of extremely poor residing in rural areas \small\coloredcite{}{r4}, climate change \small\coloredcite{}{r5} is a risk to their livelihoods. And rural households are \small\coloredcite{}{r6} susceptible to migration due to environmental, political, and economic factors \small\coloredcite{}{r7}.
			\end{justify}
			\item \begin{justify}
				\small Falling rural population and resulting migration to urban areas led to uneven distribution of resources, creating competition in urban areas, while opportunities and resources in rural areas are wasted, leading \small\coloredcite{}{r8} to scarcity of resources in urban areas, according to various sources resources.
			\end{justify}
			\item \begin{justify}
				\small Studying rural populations’ livelihoods is essential to address poverty and provide safety nets during economic shortfalls as they heavily depend on natural resources for income, according to \small\coloredcite{}{r9} and \small\coloredcite{}{r10}.
			\end{justify}
		\end{itemize}
	\end{frame}
	
	\begin{frame}[t]{Background of the study}
		\begin{center}        \includegraphics[width=0.9\paperwidth,height=0.8\paperheight]{Envrionmental dependency.pdf}
		\end{center}
	\end{frame}
	
	\begin{frame}[t]{Background of the study}
		\begin{itemize}
			\item \begin{justify} \small Nawrotzki  et. al (2012) \small\coloredcite{}{r9} underscores the central importance of natural resources in rural livelihoods. 
			\end{justify}
			\item \begin{justify}
				\small Charlery \& Walelign (2015) \small\coloredcite{}{r11} reveals that those classified as income poor exhibit a higher dependence on environmental resource.
			\end{justify}
			\item \begin{justify}
				\small Walelign & Jiao (2017) employs clustering to identify various remunerative strategy groups, highlighting a higher reliance on environmental resources within the cluster employing the least remunerative strategies \small\coloredcite{}{r12}.
			\end{justify}
			\item \begin{justify}
				\small In the study conducted by Walelign et al. (2020), it is observed that households with high reliance on environmental resources exhibit lower income and asset endowments \small\coloredcite{}{r13}.
			\end{justify}
			\item \begin{justify}
				\small Furthermore, Chhetri et al. (2022) concludes that forest and environmental income remain the primary source of income and livelihoods for poor and marginalized households, with a  notable decrease in forest and environmental incomes as household income increases \small\coloredcite{}{r14}. 
			\end{justify}
		\end{itemize}
	\end{frame}
	
	\section{Statement of the Problem}
	\begin{frame}[t]{Statement of the Problem}
		\begin{itemize}
			\item \begin{justify} 
				\small Many rural households in developing countries heavily rely on natural resources and the environment (Adger, \small\coloredcite{}{r15}; Ahmadpour et al., 2020a \small\coloredcite{}{r16}; Chambers & Conway, 1992\small\coloredcite{}{r17}; Ellis, 1999 \small\coloredcite{}{r18}), but past literature has not thoroughly studied environmental dependency and its influence on household vulnerability.  
			\end{justify} 
			
			\item \begin{justify} 
				\small Moreover, although (Mao et al., 2020 \small\coloredcite{}{r19}; Shan & Ahmed, 2020\small\coloredcite{}{r20}; Lorato, 2019) have conducted some analysis on the determinants of rural livelihood strategies. 
			\end{justify}
			
			\item \begin{justify} 
				\small Past literature has not thoroughly studied environmental dependency and its influence on household vulnerability. This could hinder the provision of a more accurate representation of environmental dependency, potentially limiting our understanding of the true impact of environmental changes on rural household vulnerability. \end{justify}  
		\end{itemize}
	\end{frame}
	
	\section{Research Question}
	\begin{frame}[t]
		\frametitle{Research Question}             
		
		\begin{block}{RQ1:}
			\begin{itemize}	
				\item What are the differences in household vulnerability across households and regions in rural Nepal?
			\end{itemize}
		\end{block} 
		\begin{exampleblock}{RQ2:}
			\begin{itemize}	
				\item What are the factors that play a role in determining the vulnerability of rural households and regions in Nepal?
			\end{itemize}
		\end{exampleblock}
	\end{frame}
	
	\section{Research Objectives}
	\begin{frame}[t]
		\frametitle{Research Objectives}             
		
		\begin{block}{General Objectives:}
			\begin{itemize}	
				\item \small Provide an overview of rural household 
				vulnerability across households.
				\item \small Determine the factors that affect the household vulnerability with a focus on rural household’s reliance on environmental resources.
			\end{itemize}
		\end{block} 
		\begin{exampleblock}{Specific Objectives:}
			\begin{itemize}	
				\item \small Compare the extent of rural household vulnerability within and across the households in selected villages from rural Nepal. Analyze the differences in the degree of vulnerability among these strategies.
				\item \small Identify and analyze the primary determinants 
				influencing the rural household vulnerability in Nepal, with a particular emphasis on the environmental reliance.
			\end{itemize}
		\end{exampleblock}
	\end{frame}
	
	
	\section{A brief review of Literature}
	\begin{frame}[allowframebreaks]
		\frametitle{A brief review of Literature}
		\textbf{\textit{Theoritical Review:}}
		\begin{itemize}
			\item \begin{justify} \small The main economic theory to study the sustainable livelihoods was developed by Robert Chambers and Gordon Conway in mid 1980s. The author created the Sustainable Livelihood Approach (SLA) to evaluate various vulnerability contexts to improve the effectiveness of development cooperation.
			\end{justify}
			
			\item \begin{justify} \small Chambers \& Conway (1992) contends that a sustainable livelihood can withstand stress and shock, maintain or improve its assets and capabilities, and create opportunities for future generations to live sustainably\small\coloredcite{}{r21}.
			\end{justify}
			\item \begin{justify} \small Some of the well-known livelihood frameworks are those proposed by the Department of International Department DfID (1999) \small\coloredcite{}{r22}, Ellis (1999) \small\coloredcite{}{r23} and Scoones (2013)\small\coloredcite{}{r24}.
			\end{justify}
		\end{itemize}
	\end{frame}
	\begin{frame}[t]{A brief review of Literature II }
		\textbf{\textit{Household Vulnerability:}}
		\begin{itemize}
			\item \begin{justify} \small Zhang et al. (2020), a study conducted in China suggests that all forms of capital (financial, human, natural, physical, and social capital) of a household were important determinants of household vulnerability \small\coloredcite{}{r25}.
			\end{justify}
			\item \begin{justify} \small Gaisie et al. (2021) find that Ghanian household capitals indicating higher economic status were linked to worse impacts from flooding but were essential for facilitating household recovery over time \small\coloredcite{}{r26}.\end{justify}
			\item \small Rahman et al. (2023) quantify cyclone vulnerability in rural Bangladesh, emphasizing the multidimensional nature of vulnerability encompassing social, economic, physical, institutional, environmental, and attitudinal factors. The research reveals distinct vulnerability patterns, particularly in environmental and composite aspects\small\coloredcite{}{r27}.
			
			\item \small Notenbaert et al. (2013) study suggests that the distance to paved roads, income diversification, and savings of households significantly influence household vulnerability \small\coloredcite{}{r28}.
		\end{itemize}
	\end{frame}
	
	\begin{frame}[t]{A brief review of Literature III }
		\textbf{\textit{Review of National Studies:}}
		\begin{itemize}
			\item \begin{justify} \small Aksha et al. (2019) investigated social vulnerability in Nepal by adapting Social Vulnerability Index (SoVI) methods to the Nepalese context using the full data set of the 2011 census provided by the Central Bureau of Statistics (CBS). \small\coloredcite{}{r29}.
			\end{justify}
			\item \begin{justify} \small Shahi & Shreezal (2020) estimate the vulnerability score for Nepal using a three-stage feasible generalized least square technique to assess the vulnerability to poverty. Using the third round of Nepal Living Standard Survey data. \small\coloredcite{}{r26}.\end{justify}
			
			\item \small On a household level, Bista (2019) examines the relationship between the magnitude of climate variability and household vulnerability in the catchment areas of Sot Khola sub-water basin in the western mountainous Surkhet, Nepal.\small\coloredcite{}{r31}.
			
			\item \small A study by Gerlitz et al. (2017) based on Hindu Kush Himalayas (HKH) collects data from 2311 households from six districts (Khotang, Udaypur, Siraha, Dolakha, Sunsari, Kavrepalanchok) in Koshi sub-basin in Nepal and computes the Multidimensional Livelihood Index (MLVI). \small\coloredcite{}{r32}
		\end{itemize}
	\end{frame}
	
	
	\section{Research Gap}
	\begin{frame}
		\frametitle{Research Gap}
		\vfill
		\begin{block}{RG:}
			\begin{itemize}
				\item \begin{justify} Scant literature on the  
					association of Environmental Dependency and Household Vulnerability. \end{justify} 
				
				\item \begin{justify} Predominant reliance on cross-sectional data in the assessment of Household Vulnerability. \end{justify} 
				
				\item \begin{justify} Fill gap: utlization of longitudinal data to capture the intertemporal dynamics of the household vulnerability in the diverse physio-graphic regions on Nepal.\end{justify} 
				
			\end{itemize}
		\end{block}
		\vfill
	\end{frame}
	
	\section{Research Methodology}
	\begin{frame}
		\frametitle{Research Design}
		\begin{exampleblock}{RD:}
			\begin{itemize}
				\item Descriptive and analytical
				\item Construction of Household Vulnerability Index based on various capitals owned by the households
				\item Environmental Dependency and Household Vulnerability
			\end{itemize}
		\end{exampleblock}
	\end{frame}
	
	\begin{frame}
		\frametitle{Conceptual Framework}
		\begin{exampleblock}{\centering Objective 1:Household Vulnerability Index}
		\end{exampleblock}
		\includegraphics[width=0.9\paperwidth,height=0.7\paperheight]{Objective 1.pdf}
	\end{frame}
	\begin{frame}
		\frametitle{Conceptual Framework}
		\begin{exampleblock}{\centering Objective 2:Household Vulnerability and Environmental Dependence}
		\end{exampleblock}
		\includegraphics[width=0.9\paperwidth,height=0.65\paperheight]{Objective 2.pdf}
	\end{frame}
	
	\begin{frame}
		\frametitle{Sources of Data}
		\begin{exampleblock}{Sources of Data:}
			\begin{itemize}
				\item Poverty and Environment Network (PEN) Data set 
			\end{itemize}
		\end{exampleblock}
		\begin{center}
			\includegraphics[width=0.7\paperwidth,height=0.55\paperheight]{Study site.PNG}
		\end{center}
	\end{frame}
	\begin{frame}
		\frametitle{Variables}
		\begin{exampleblock}{Variables used in HVI Construction}
		\end{exampleblock}
		\begin{center}
			\includegraphics[width=0.8\paperwidth,height=0.65\paperheight]{Variables used in HVI.pdf}
		\end{center}
	\end{frame}
	
	\begin{frame}
		\frametitle{Variables}
		\begin{exampleblock}{Factors affecting HVI}
		\end{exampleblock}
		\vspace{-65pt}
		\begin{center}
			\includegraphics[width=0.80\paperwidth,height=0.75\paperheight]{Factors affecting HVI.pdf}
		\end{center}
	\end{frame}
	
	\begin{frame}
		\frametitle{Techniques of Data Analysis}
		\begin{block}{Household Vulnerability Index Construction}
			\textbf{\textit{Mini-Max Method:}}
			\begin{align}
				\text{X}_{\text{ij}} &= \frac{X_{\text{i}} - X_{\text{Minj}}}{X_{\text{Maxj}} - X_{\text{Minj}}} \tag{1} \\[0.5cm] 	
				\text{X}_{\text{ij}} &= \frac{X_{\text{Maxj}} - X_{\text{i}}}{X_{\text{Maxj}} - X_{\text{Minj}}} \tag{2}\\[0.5cm]
				\text{HVI}_{\text{Cij}} &= \frac{1}{n}\sum_{i=1}^{n}\text{X}_{\text{ij}} \tag{3}\\[0.5cm]
				\text{HVI}_{\text{ij}} &= \sum_{i=1}^{n}\text{HVI}_{\text{Cij}} \tag{4}
			\end{align} 
		\end{block}
	\end{frame}
	
	\begin{frame}
		\frametitle{Techniques of Data Analysis}
		\begin{block}{Household Vulnerability and Environmental Dependence}
			\textbf{\textit{Pooled OLS, Fixed Effect, Random Effect}}
			\begin{align}
				\mathit{HVI}_{i,t} &= \beta_{0} + \delta \mathit{\mathbf{ED}_{i,t}} + \mathbf{X}_{i,t} \beta + \mathbf{Z}_i \lambda + \mathbf{T}_t \delta_t + \boldsymbol{\epsilon}\tag{5}
			\end{align}
			Where,  
			\\
			\textit{HVI \ \ \ \ \ \ \ \ \ \ \ \ \ \ \ \ \ \ = Household Vulnerability Index}\\
			
			\textit{ED \ \ \ \ \ \ \ \ \ \ \ \ \ \ \ \ \ \ \ \ = Environmental Dependence}\\
			
			\textit{X \ \ \ \ \ \ \ \ \ \ \ \ \ \ \ \ \ \ \ \ \ \ \ = Controls (Dependency ratio, Debt, Shocks)}\\
			
			\textit{Z \ \ \ \ \ \ \ \ \ \ \ \ \ \ \ \ \ \ \ \ \ \ \  = Time invariant fixed effects (District and VDC)}\\
			
			\textit{T \ \ \ \ \ \ \ \ \ \ \ \ \ \ \ \ \ \ \ \ \ \ \ = Year control variable}\\
			
			\textit{ \beta_{0}, \delta, \beta, \lambda, \delta_t   \ \ \ \ = Parameters }\\
			\vspace{0.5cm}
		\end{block} 
	\end{frame}      
	
	\begin{frame}{Techniques of Data Analysis}
		\begin{exampleblock}{Diagnostic Tests}
			\begin{itemize}
				\item Individual Effects Test
				\item Time Fixed Effects Test
				\item  Breusch-Pagan Lagrange Multiplier (BPLM) Test
				\item  Hausman Specification Test
			\end{itemize}
		\end{exampleblock}
	\end{frame}
	
	\section{Results and Discussions}
	\begin{frame}{Descriptive Statistics}
		\begin{block}{\centering Household Vulnerability Index}
			\begin{center}
				\begin{landscape}
					\begin{table}
						\renewcommand{\arraystretch}{1}
						\resizebox{1\textwidth}{!}{%
							\begin{tabular}{lccccccccc} \hline
								\textbf{Year}                    & \multicolumn{3}{c}{\textbf{2006}}                                                                                                                                                                    & \multicolumn{3}{c}{\textbf{2009}}                                                                                                                                                                    & \multicolumn{3}{c}{\textbf{2012}}                                                                                                                                                                    \\ \hline
								\textbf{District}                & \textbf{Chitwan}                                                & \textbf{Kaski}                                                  & \textbf{Mustang}                                                 & \textbf{Chitwan}                                                & \textbf{Kaski}                                                  & \textbf{Mustang}                                                 & \textbf{Chitwan}                                                & \textbf{Kaski}                                                  & \textbf{Mustang}                                                 \\ \hline
								\textbf{Human Capital}           &                                                                 &                                                                 &                                                                  &                                                                 &                                                                 &                                                                  &                                                                 &                                                                 &                                                                  \\ 
								hhh\_age                         & \begin{tabular}[c]{@{}c@{}}50.36 \\ (14.15)\end{tabular}        & \begin{tabular}[c]{@{}c@{}}50.14 \\ (14.57)\end{tabular}        & \begin{tabular}[c]{@{}c@{}}52.89  \\ (13.52)\end{tabular}        & \begin{tabular}[c]{@{}c@{}}52.13  \\ (13.75)\end{tabular}       & \begin{tabular}[c]{@{}c@{}}52.00  \\ (13.39)\end{tabular}       & \begin{tabular}[c]{@{}c@{}}54.12  \\ (13.78)\end{tabular}        & \begin{tabular}[c]{@{}c@{}}52.24  \\ (17.20)\end{tabular}       & \begin{tabular}[c]{@{}c@{}}53.52  \\ (13.71)\end{tabular}       & \begin{tabular}[c]{@{}c@{}}55.24  \\ (14.17)\end{tabular}        \\
								hhh\_edu                         & \begin{tabular}[c]{@{}c@{}}3.08  \\ (4.06)\end{tabular}         & \begin{tabular}[c]{@{}c@{}}6.29  \\ (4.97)\end{tabular}         & \begin{tabular}[c]{@{}c@{}}3.05  \\ (3.98)\end{tabular}          & \begin{tabular}[c]{@{}c@{}}2.93  \\ (4.06)\end{tabular}         & \begin{tabular}[c]{@{}c@{}}6.07  \\ (5.23)\end{tabular}         & \begin{tabular}[c]{@{}c@{}}2.94  \\ (3.78)\end{tabular}          & \begin{tabular}[c]{@{}c@{}}2.91  \\ (4.33)\end{tabular}         & \begin{tabular}[c]{@{}c@{}}6.91  \\ (5.07)\end{tabular}         & \begin{tabular}[c]{@{}c@{}}2.90  \\ (4.08)\end{tabular}          \\
								max\_hh\_edu                     & \begin{tabular}[c]{@{}c@{}}8.44  \\ (3.91)\end{tabular}         & \begin{tabular}[c]{@{}c@{}}10.76  \\ (2.90)\end{tabular}        & \begin{tabular}[c]{@{}c@{}}7.64  \\ (3.32)\end{tabular}          & \begin{tabular}[c]{@{}c@{}}9.70  \\ (3.63)\end{tabular}         & \begin{tabular}[c]{@{}c@{}}11.18  \\ (3.94)\end{tabular}        & \begin{tabular}[c]{@{}c@{}}8.04  \\ (3.93)\end{tabular}          & \begin{tabular}[c]{@{}c@{}}9.89  \\ (4.44)\end{tabular}         & \begin{tabular}[c]{@{}c@{}}11.91 \\  (4.03)\end{tabular}        & \begin{tabular}[c]{@{}c@{}}8.22  \\ (3.87)\end{tabular}          \\
								\textbf{Physical Capital}        &                                                                 &                                                                 &                                                                  &                                                                 &                                                                 &                                                                  &                                                                 &                                                                 &                                                                  \\
								implements                       & \begin{tabular}[c]{@{}c@{}}4660.32  \\ (11275.51)\end{tabular}  & \begin{tabular}[c]{@{}c@{}}14057.03  \\ (16860.40)\end{tabular} & \begin{tabular}[c]{@{}c@{}}10360.32  \\ (19629.35)\end{tabular}  & \begin{tabular}[c]{@{}c@{}}10153.80  \\ (23970.96)\end{tabular} & \begin{tabular}[c]{@{}c@{}}30700.04  \\ (46128.42)\end{tabular} & \begin{tabular}[c]{@{}c@{}}15135.16  \\ (25508.99)\end{tabular}  & \begin{tabular}[c]{@{}c@{}}22165.29  \\ (38089.26)\end{tabular} & \begin{tabular}[c]{@{}c@{}}48959.03  \\ (70582.61)\end{tabular} & \begin{tabular}[c]{@{}c@{}}21466.58  \\ (27566.06)\end{tabular}  \\
								livestock                        & \begin{tabular}[c]{@{}c@{}}18532.68  \\ (15428.31)\end{tabular} & \begin{tabular}[c]{@{}c@{}}26573.08  \\ (20411.58)\end{tabular} & \begin{tabular}[c]{@{}c@{}}80387.77  \\ (224589.10)\end{tabular} & \begin{tabular}[c]{@{}c@{}}43936.83  \\ (39679.86)\end{tabular} & \begin{tabular}[c]{@{}c@{}}35690.11  \\ (35760.04)\end{tabular} & \begin{tabular}[c]{@{}c@{}}56165.26  \\ (178639.73)\end{tabular} & \begin{tabular}[c]{@{}c@{}}38993.71  \\ (34330.39)\end{tabular} & \begin{tabular}[c]{@{}c@{}}34635.85  \\ (39306.64)\end{tabular} & \begin{tabular}[c]{@{}c@{}}34114.52  \\ (39335.73)\end{tabular}  \\
								land                             & \begin{tabular}[c]{@{}c@{}}2027.47  \\ (6367.27)\end{tabular}   & \begin{tabular}[c]{@{}c@{}}1187.00  \\ (1013.02)\end{tabular}   & \begin{tabular}[c]{@{}c@{}}2940.39  \\ (2789.36)\end{tabular}    & \begin{tabular}[c]{@{}c@{}}915.91  \\ (765.38)\end{tabular}     & \begin{tabular}[c]{@{}c@{}}1491.41  \\ (2060.26)\end{tabular}   & \begin{tabular}[c]{@{}c@{}}2235.09  \\ (3738.40)\end{tabular}    & \begin{tabular}[c]{@{}c@{}}1041.46  \\ (1136.88)\end{tabular}   & \begin{tabular}[c]{@{}c@{}}1374.96  \\ (2253.95)\end{tabular}   & \begin{tabular}[c]{@{}c@{}}1921.22  \\ (1892.77)\end{tabular}    \\
								\textbf{Social Capital}          &                                                                 &                                                                 &                                                                  &                                                                 &                                                                 &                                                                  &                                                                 &                                                                 &                                                                  \\
								hh\_caste                        & \begin{tabular}[c]{@{}c@{}}0.58  \\ (0.50)\end{tabular}         & \begin{tabular}[c]{@{}c@{}}0.89  \\ (0.32)\end{tabular}         & \begin{tabular}[c]{@{}c@{}}0.49  \\ (0.50)\end{tabular}          & \begin{tabular}[c]{@{}c@{}}0.66  \\ (0.48)\end{tabular}         & \begin{tabular}[c]{@{}c@{}}0.98  \\ (0.14)\end{tabular}         & \begin{tabular}[c]{@{}c@{}}0.58  \\ (0.50)\end{tabular}          & \begin{tabular}[c]{@{}c@{}}0.50  \\ (0.50)\end{tabular}         & \begin{tabular}[c]{@{}c@{}}0.86  \\ (0.50)\end{tabular}         & \begin{tabular}[c]{@{}c@{}}0.59 \\ (0.42)\end{tabular}           \\
								\textbf{Financial Capital}       &                                                                 &                                                                 &                                                                  &                                                                 &                                                                 &                                                                  &                                                                 &                                                                 &                                                                  \\
								bank\_saving                     & \begin{tabular}[c]{@{}c@{}}879.58  \\ (2661.50)\end{tabular}    & \begin{tabular}[c]{@{}c@{}}9663.83  \\ (26812.59)\end{tabular}  & \begin{tabular}[c]{@{}c@{}}31897.65  \\ (79933.66 )\end{tabular} & \begin{tabular}[c]{@{}c@{}}1911.63  \\ (6126.69)\end{tabular}   & \begin{tabular}[c]{@{}c@{}}11937.72  \\ (31025.90)\end{tabular} & \begin{tabular}[c]{@{}c@{}}24536.06  \\ (59338.85)\end{tabular}  & \begin{tabular}[c]{@{}c@{}}11953.55  \\ (31763.88)\end{tabular} & \begin{tabular}[c]{@{}c@{}}25410.64  \\ (66932.36)\end{tabular} & \begin{tabular}[c]{@{}c@{}}48051.24  \\ (104495.00)\end{tabular} \\
								jewellery                        & \begin{tabular}[c]{@{}c@{}}0.00  \\ (0.00)\end{tabular}         & \begin{tabular}[c]{@{}c@{}}0.00  \\ (0.00)\end{tabular}         & \begin{tabular}[c]{@{}c@{}}31662.91  \\ (67846.57)\end{tabular}  & \begin{tabular}[c]{@{}c@{}}4396.88  \\ (6965.54)\end{tabular}   & \begin{tabular}[c]{@{}c@{}}20485.87  \\ (16594.27)\end{tabular} & \begin{tabular}[c]{@{}c@{}}38598.35  \\ (79440.96)\end{tabular}  & \begin{tabular}[c]{@{}c@{}}21477.20  \\ (23620.48)\end{tabular} & \begin{tabular}[c]{@{}c@{}}51605.95  \\ (48463.66)\end{tabular} & \begin{tabular}[c]{@{}c@{}}54132.35  \\ (112328.06)\end{tabular} \\
								\textbf{Livelihood}              &                                                                 &                                                                 &                                                                  &                                                                 &                                                                 &                                                                  &                                                                 &                                                                 &                                                                  \\
								n\_livelihoods                   & \begin{tabular}[c]{@{}c@{}}4.81\\ (0.97)\end{tabular}           & \begin{tabular}[c]{@{}c@{}}4.72\\ (0.91)\end{tabular}           & \begin{tabular}[c]{@{}c@{}}4.56\\ (0.93)\end{tabular}            & \begin{tabular}[c]{@{}c@{}}4.93\\ (1.02)\end{tabular}           & \begin{tabular}[c]{@{}c@{}}4.74\\ (0.91)\end{tabular}           & \begin{tabular}[c]{@{}c@{}}5.11\\ (0.84)\end{tabular}            & \begin{tabular}[c]{@{}c@{}}4.60\\ (0.98)\end{tabular}           & \begin{tabular}[c]{@{}c@{}}4.78\\ (0.90)\end{tabular}           & \begin{tabular}[c]{@{}c@{}}4.60\\ (0.98)\end{tabular}            \\
								\textbf{Household Vulnerability} &                                                                 &                                                                 &                                                                  &                                                                 &                                                                 &                                                                  &                                                                 &                                                                 &                                                                  \\
								HVI                              & \begin{tabular}[c]{@{}c@{}}0.62  \\ (0.05)\end{tabular}         & \begin{tabular}[c]{@{}c@{}}0.62  \\ (0.04)\end{tabular}         & \begin{tabular}[c]{@{}c@{}}0.64  \\ (0.05)\end{tabular}          & \begin{tabular}[c]{@{}c@{}}0.61  \\ (0.05)\end{tabular}         & \begin{tabular}[c]{@{}c@{}}0.62  \\ (0.04)\end{tabular}         & \begin{tabular}[c]{@{}c@{}}0.63  \\ (0.05)\end{tabular}          & \begin{tabular}[c]{@{}c@{}}0.61  \\ (0.05)\end{tabular}         & \begin{tabular}[c]{@{}c@{}}0.62  \\ (0.05)\end{tabular}         & \begin{tabular}[c]{@{}c@{}}0.63  \\ (0.05)\end{tabular}         \\ \hline \hline
							\end{tabular}
						}
					\end{table}
				\end{landscape}
			\end{center}    
		\end{block}
	\end{frame}
	
	
	
	
	\begin{frame}[t]{Descriptive Statistics}
		\begin{exampleblock}{\centering Household Vulenrability and Environmental Dependence}
			\begin{table}[H]
				\renewcommand{\arraystretch}{0.7}
				\resizebox{0.7\linewidth}{!}{
					\begin{center}
						\begin{tabular}{llcccc} \hline
							\multicolumn{2}{l}{\textbf{Year}}     & \multicolumn{4}{c}{\textbf{2006}}                                                                                                                                                                                                                                 \\ \hline
							\multicolumn{2}{l}{\textbf{District}} & \textbf{Chitwan}                                               & \textbf{Kaski}                                                 & \textbf{Mustang}                                               & \textbf{Mustang}                                               \\ \hline
							\multicolumn{2}{l}{\textbf{VDC}}      & \textbf{Chainpur}                                              & \textbf{Hemja}                                                 & \textbf{Kunjo}                                                 & \textbf{Lete}                                                  \\ \hline
							\multicolumn{2}{l}{env\_dependence}   & \begin{tabular}[c]{@{}c@{}}0.13\\      (0.19)\end{tabular}     & \begin{tabular}[c]{@{}c@{}}0.16\\      (0.20)\end{tabular}     & \begin{tabular}[c]{@{}c@{}}0.40\\      (0.23)\end{tabular}     & \begin{tabular}[c]{@{}c@{}}0.29\\      (0.23)\end{tabular}     \\
							\multicolumn{2}{l}{dependency\_ratio} & \begin{tabular}[c]{@{}c@{}}0.66\\      (0.62)\end{tabular}     & \begin{tabular}[c]{@{}c@{}}0.73\\      (0.76)\end{tabular}     & \begin{tabular}[c]{@{}c@{}}0.88\\      (0.85)\end{tabular}     & \begin{tabular}[c]{@{}c@{}}0.69\\      (0.63)\end{tabular}     \\
							\multicolumn{2}{l}{debt}              & \begin{tabular}[c]{@{}c@{}}12234.42\\ (19246.56)\end{tabular}  & \begin{tabular}[c]{@{}c@{}}25896.24\\ (47371.72)\end{tabular}  & \begin{tabular}[c]{@{}c@{}}17792.57\\ (19055.29)\end{tabular}  & \begin{tabular}[c]{@{}c@{}}31217.37\\  (69576.58)\end{tabular} \\
							\multicolumn{2}{l}{shock}             & \begin{tabular}[c]{@{}c@{}}1.75\\      (1.66)\end{tabular}     & \begin{tabular}[c]{@{}c@{}}0.55\\      (1.07)\end{tabular}     & \begin{tabular}[c]{@{}c@{}}2.48\\      (1.24)\end{tabular}     & \begin{tabular}[c]{@{}c@{}}1.97\\      (1.2)\end{tabular}      \\ \hline
							\multicolumn{2}{l}{\textbf{Year}}     & \multicolumn{4}{c}{\textbf{2009}}                                                                                                                                                                                                                                 \\ \hline
							\multicolumn{2}{l}{env\_dependence}   & \begin{tabular}[c]{@{}c@{}}0.14\\      (0.23)\end{tabular}     & \begin{tabular}[c]{@{}c@{}}0.13\\      (0.15)\end{tabular}     & \begin{tabular}[c]{@{}c@{}}0.19\\      (0.30)\end{tabular}     & \begin{tabular}[c]{@{}c@{}}0.21\\      (0.59)\end{tabular}     \\
							\multicolumn{2}{l}{dependency\_ratio} & \begin{tabular}[c]{@{}c@{}}0.58\\      (0.56)\end{tabular}     & \begin{tabular}[c]{@{}c@{}}0.63\\      (0.63)\end{tabular}     & \begin{tabular}[c]{@{}c@{}}0.77\\      (0.64)\end{tabular}     & \begin{tabular}[c]{@{}c@{}}0.68\\      (0.69)\end{tabular}     \\
							\multicolumn{2}{l}{debt}              & \begin{tabular}[c]{@{}c@{}}18249.69\\  (36642.37)\end{tabular} & \begin{tabular}[c]{@{}c@{}}46654.49\\  (72705.08)\end{tabular} & \begin{tabular}[c]{@{}c@{}}14887.21\\  (16156.9)\end{tabular}  & \begin{tabular}[c]{@{}c@{}}20616.94\\ (41324.71)\end{tabular}  \\
							\multicolumn{2}{l}{shock}             & \begin{tabular}[c]{@{}c@{}}0.24\\      (0.65)\end{tabular}     & \begin{tabular}[c]{@{}c@{}}0.59\\      (0.96)\end{tabular}     & \begin{tabular}[c]{@{}c@{}}0.65\\      (1.00)\end{tabular}     & \begin{tabular}[c]{@{}c@{}}1.16\\      (1.14)\end{tabular}     \\ \hline
							\multicolumn{2}{l}{\textbf{Year}}     & \multicolumn{4}{c}{\textbf{2012}}                                                                                                                                                                                                                                 \\ \hline
							\multicolumn{2}{l}{env\_dependence}   & \begin{tabular}[c]{@{}c@{}}0.15\\      (0.20)\end{tabular}     & \begin{tabular}[c]{@{}c@{}}0.14\\      (0.25)\end{tabular}     & \begin{tabular}[c]{@{}c@{}}0.78\\      (3.16)\end{tabular}     & \begin{tabular}[c]{@{}c@{}}0.27\\      (0.25)\end{tabular}     \\
							\multicolumn{2}{l}{dependency\_ratio} & \begin{tabular}[c]{@{}c@{}}0.51\\      (0.60)\end{tabular}     & \begin{tabular}[c]{@{}c@{}}0.53\\      (0.57)\end{tabular}     & \begin{tabular}[c]{@{}c@{}}0.77\\      (0.82)\end{tabular}     & \begin{tabular}[c]{@{}c@{}}0.49\\      (0.56)\end{tabular}     \\
							\multicolumn{2}{l}{debt}              & \begin{tabular}[c]{@{}c@{}}37221.48\\ (87094.30)\end{tabular}  & \begin{tabular}[c]{@{}c@{}}64572.68\\ (133587.15)\end{tabular} & \begin{tabular}[c]{@{}c@{}}24507.84\\  (33242.18)\end{tabular} & \begin{tabular}[c]{@{}c@{}}19864.55\\ (27238.68)\end{tabular}  \\
							\multicolumn{2}{l}{shock}             & \begin{tabular}[c]{@{}c@{}}0.77\\      (1.01)\end{tabular}     & \begin{tabular}[c]{@{}c@{}}0.31\\      (0.66)\end{tabular}     & \begin{tabular}[c]{@{}c@{}}0.35\\      (0.72)\end{tabular}     & \begin{tabular}[c]{@{}c@{}}0.24\\      (0.59)\end{tabular}  \\ \hline \hline  
						\end{tabular} 
					\end{center}
				}
			\end{table}
		\end{exampleblock}
	\end{frame}
	
	\begin{frame}{Radar Chart}
		\begin{block}{District level Household Vulnerability}
			\begin{table}[H]
				\resizebox{\textwidth}{!}{%
					\begin{tabular}{cccccccccc} \hline
						\multirow{2}{*}{\textbf{\begin{tabular}[c]{@{}c@{}}Year/\\ District\end{tabular}}} & \multicolumn{3}{c}{\textbf{2006}}                                                                                                                                     & \multicolumn{3}{c}{\textbf{2009}}                                                                                                                                     & \multicolumn{3}{c}{\textbf{2012}}                                                                                                                                     \\ \hline
						& \textbf{Chitwan}                                      & \textbf{Kaski}                                        & \textbf{Mustang}                                      & \textbf{Chitwan}                                      & \textbf{Kaski}                                        & \textbf{Mustang}                                      & \textbf{Chitwan}                                      & \textbf{Kaski}                                        & \textbf{Mustang}                                      \\ \hline
						\textbf{HVI}                                                                       & \begin{tabular}[c]{@{}c@{}}0.61\\ (0.05)\end{tabular} & \begin{tabular}[c]{@{}c@{}}0.62\\ (0.04)\end{tabular} & \begin{tabular}[c]{@{}c@{}}0.64\\ (0.05)\end{tabular} & \begin{tabular}[c]{@{}c@{}}0.61\\ (0.05)\end{tabular} & \begin{tabular}[c]{@{}c@{}}0.61\\ (0.04)\end{tabular} & \begin{tabular}[c]{@{}c@{}}0.63\\ (0.05)\end{tabular} & \begin{tabular}[c]{@{}c@{}}0.61\\ (0.05)\end{tabular} & \begin{tabular}[c]{@{}c@{}}0.62\\ (0.05)\end{tabular} & \begin{tabular}[c]{@{}c@{}}0.63\\ (0.05)\end{tabular} \\ \hline \hline
					\end{tabular}
				}
			\end{table}
		\end{block}
		\vspace{-165pt}
		\begin{center}
			\includegraphics[width=1.12\textwidth]{HVI_Summary_District_Panel25012024.pdf}
		\end{center}
	\end{frame}
	
	\begin{frame}{Radar Chart}
		\begin{block}{Village level Household Vulnerability}
			\begin{table}[H]
				\resizebox{\textwidth}{!}{%
					\begin{tabular}{cccclccclcccl} \hline
						\multirow{2}{*}{\textbf{\begin{tabular}[c]{@{}c@{}}Year/\\ District\end{tabular}}} & \multicolumn{4}{c}{\textbf{2006}}                                                                                                                                                                                             & \multicolumn{4}{c}{\textbf{2009}}                                                                                                                                                                                             & \multicolumn{4}{c}{\textbf{2012}}                                                                                                                                                                                             \\ \hline
						& \textbf{Chainpur}                                     & \textbf{Hemja}                                        & \textbf{Kunjo}                                        & \textbf{Lete}                                         & \textbf{Chainpur}                                     & \textbf{Hemja}                                        & \textbf{Kunjo}                                        & \textbf{Lete}                                         & \textbf{Chainpur}                                     & \textbf{Hemja}                                        & \textbf{Kunjo}                                        & \textbf{Lete}                                         \\ \hline
						\textbf{HVI}                                                                       & \begin{tabular}[c]{@{}c@{}}0.61\\ (0.05)\end{tabular} & \begin{tabular}[c]{@{}c@{}}0.62\\ (0.04)\end{tabular} & \begin{tabular}[c]{@{}c@{}}0.65\\ (0.04)\end{tabular} & \begin{tabular}[c]{@{}l@{}}0.63\\ (0.05)\end{tabular} & \begin{tabular}[c]{@{}c@{}}0.61\\ (0.05)\end{tabular} & \begin{tabular}[c]{@{}c@{}}0.61\\ (0.04)\end{tabular} & \begin{tabular}[c]{@{}c@{}}0.64\\ (0.04)\end{tabular} & \begin{tabular}[c]{@{}l@{}}0.63\\ (0.05)\end{tabular} & \begin{tabular}[c]{@{}c@{}}0.61\\ (0.05)\end{tabular} & \begin{tabular}[c]{@{}c@{}}0.62\\ (0.05)\end{tabular} & \begin{tabular}[c]{@{}c@{}}0.63\\ (0.05)\end{tabular} & \begin{tabular}[c]{@{}l@{}}0.61\\ (0.05)\end{tabular} \\ \hline \hline
					\end{tabular}
				}
			\end{table}
		\end{block}
		\vspace{-35pt}
		\begin{center}
			\includegraphics[width=1\textwidth]{HVI_Summary_VDC_25012024.pdf}
		\end{center}
	\end{frame}
	
	\begin{frame}{Persistence and Transience of Household Vulnerability}
		\begin{exampleblock}{Vulnerability level across the waves of survey}
			\begin{table}[H]
				\begin{center}
					\begin{tabular}{lccc} \hline
						\textbf{Vulnerability level} & \textbf{2006}            & \textbf{2009}            & \textbf{2012}           \\ \hline
						High                         & 55                       & 43                       & 41                      \\
						Moderate                     & 302                      & 289                      & 297                     \\
						Low                          & 71                       & 96                       & 90                \\ \hline \hline     
					\end{tabular}\\
				\end{center}
			\end{table}
		\end{exampleblock}
		
		\begin{alertblock}{Persistence of Vulnerability of the Households 2006-2009-2012}
			\begin{table}[H]
				\begin{center}
					\begin{tabular}{lc} \hline
						\textbf{Vulnerability level} & \textbf{No. of Households} \\	\hline
						High to High & 6 \\
						Low to Low            & 18         \\
						Moderate to Moderate  & 163  \\ \hline \hline     
					\end{tabular} \\
				\end{center}
			\end{table}
		\end{alertblock}
	\end{frame}
	
	\begin{frame}{Persistence and Transience of Household Vulnerability}
		\begin{block}{Transition Matrix of the Household Vulnerability}
			\begin{table}[H]
				\begin{center}
					\begin{tabular}{lcc} \hline
						\textbf{Vulnerability Level} & \textbf{2006-2009} & \textbf{2009-2012} \\ \hline
						High to High                 & 12                 & 11                 \\
						High to low                  & 7                  & 2                  \\
						High to Moderate             & 36                 & 30                 \\
						Low to High                  & 2                  & 3                  \\
						Low to Low                   & 36                 & 41                 \\
						Low to moderate              & 33                 & 52                 \\
						Moderate to High             & 29                 & 27                 \\
						Moderate to low              & 53                 & 47                 \\
						Moderate to Moderate         & 220                & 215       \\ \hline \hline        
					\end{tabular} \\
				\end{center}
			\end{table}
		\end{block}
	\end{frame}
	
	\begin{frame}{Persistence and Transience of Household Vulnerability}
		\begin{exampleblock}{}
			\begin{center}
				\includegraphics[width=1\textwidth]{Sankey.pdf}
			\end{center} 
		\end{exampleblock}
	\end{frame}
	
	\begin{frame}{Results from Empirical Analysis}
		\begin{exampleblock}{Panel Data Regression}
			\begin{center}
				\includegraphics[width=0.7\paperwidth,height=0.7\paperheight]{Panel Data Regression.png}
			\end{center} 
		\end{exampleblock}
	\end{frame}
	
	\begin{frame}{Results from Empirical Analysis}
		\begin{exampleblock}{Pooled OLS Regression}
			\begin{center}
				\includegraphics[width=0.7\paperwidth,height=0.7\paperheight]{PooledOLS.png}
			\end{center} 
		\end{exampleblock}
	\end{frame}
	
	\begin{frame}{Results from Empirical Analysis}
		\begin{exampleblock}{Random Effects Regression}
			\begin{center}
				\includegraphics[width=0.7\paperwidth,height=0.7\paperheight]{Random Effects.png}
			\end{center} 
		\end{exampleblock}
	\end{frame}
	
	\begin{frame}{Results from Empirical Analysis}
		\begin{exampleblock}{Fixed Effects Regression}
			\begin{center}
				\includegraphics[width=0.7\paperwidth,height=0.7\paperheight]{Fixed Effects.png}
			\end{center} 
		\end{exampleblock}
	\end{frame}
	
\begin{frame}{Diagnostic Tests}
	\begin{exampleblock}{}
		\begin{center}
			\includegraphics[width=0.7\paperwidth,height=0.7\paperheight]{Time and Individual Fixed effects.jpg\\
			}
		\end{center} 
	\end{exampleblock}
\end{frame}	
	
\begin{frame}{Diagnostic Tests}
	\begin{exampleblock}{}
		\begin{center}
			\includegraphics[width=0.7\paperwidth,height=0.7\paperheight]{BPLM Hausman.pdf\\
			}
		\end{center} 
	\end{exampleblock}
\end{frame}		
	
	\begin{frame}{Discussion}
		\begin{block}{Discussion}
			\scriptsize \begin{itemize}
				\item Environmental Dependence \& Household Vulnerability negative relationship in POLS (1), POLS (2) and RE in Table Panel Data Regression . 
				\item Positive and significant relationship of Environmental Dependence with the outcome variable, even after controlling for control variables in the step-wise Pooled OLS regression, Random Effects regression and Fixed Effects regression.
				\item Control variable Debt had a negative association with the Household Vulnerability. However, the association is not significant.
				\item Dependency ratio depicted to have significantly positive relationship with the Household Vulnerability.
				\item Shock also displayed a positive relationship with the Household Vulnerability. However, the significance was only observed in the model 4 of all models. 
				\end{itemize} 
		\end{block}
	\end{frame}
	\section{Conclusions }
	
	\begin{frame}{Conclusions}
		\begin{exampleblock}{Conclusions}  
			
			\begin{itemize} \scriptsize
				\item Positive association to Environmental Dependency to the Household Vulnerability.
				\item Increase in Environmental  Dependency, the vulnerability level increases.
				\item Debt decreases the vulnerability.
				\item Dependency ratio increases vulnerability increases.
				\item Shock increases vulnerability level.
				\item Some level of Vulnerability in all districts. However, Mountainous district depicted higher level of vulnerability.
				\item This contrast between the Nepal’s diverse geographic landscape must be taken into consideration while drafting the interventions to address specific vulnerabilities.
				\item  Policies that attempt to lower the dependency on the environment must be designed and implemented in rural setting.
				\item Additionally, community-level resilience programs, educational initiatives, and cross-sectoral collaboration could be proposed to address the multifaceted nature of household vulnerability.
				\item Foster sustainable development, resilience, and improved livelihoods in rural Nepalese communities.      
			\end{itemize}
		\end{exampleblock}
	\end{frame}
	
	\section{Reference}
	\begin{frame}[allowframebreaks]
		\frametitle{References}
		\tiny{\bibliographystyle{abbrv}}
		\scriptsize\bibliography{bibfile} 
		
	\end{frame}
	
	\begin{frame}
		\includegraphics[width=1.20\textwidth]{thank you.jpg}
	\end{frame}
	
\end{document} 